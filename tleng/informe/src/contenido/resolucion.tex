\section{Resolución}

Analizando las gramaticas del lenguaje Jay dada por la catedra y el funcionamiento de las herramientas propuestas, 
pudimos observar que la gramatica no era $LL(1)$ debido a que tenía recursión a izquierda principalmente.
Lo que decidimos hacer entonces es utilizar la herramienta $JavaCC$ que si bien solo funciona para gramaticas
$ELL$ con un pequeño pasaje donde agregaremos los simbolos $^*$, $^+$ y $?$, podremos obtener facilmente la
gramatica $ELL$ deseada.

\subsection{Transformación de la gramática del lexer}

Dados los siguientes componentes léxicos de Jay:

\noindent$InputElement$ $\rightarrow$ $WhiteSpace$ | $Comment$ | $Token$ \\
$WhiteSpace$ $\rightarrow$ {\bf space} | $\backslash$ t | $\backslash$r | $\backslash$n | $\backslash$f | $\backslash$r$\backslash$n \\
$Comment$ $\rightarrow$ // any string ended by $\backslash$r or $\backslash$n or $\backslash$r$\backslash$n \\
$Token$ $\rightarrow$ $Identifier$ | $Keyword$ | $Literal$ | $Separator$ | $Operator$ \\
$Identifier$ $\rightarrow$ $Letter$ | $Identifier$$Letter$ | $Identifier$$Digit$ \\
$Letter$ $\rightarrow$ a | b | ... | z | A | B | ... | Z \\
$Digit$ $\rightarrow$ 0 | 1 | 2 | ... | 9 \\
$Keyword$ $\rightarrow$ {\bf boolean} | {\bf else} | {\bf if} | {\bf int} | {\bf main} | {\bf void} | {\bf while} \\
$Literal$ $\rightarrow$ $Boolean$ | $Integer$ \\ 
$Boolean$ $\rightarrow$ {\bf true} | {\bf false}\\
$Integer$ $\rightarrow$ $Digit$ | $Integer$$Digit$ \\
$Separator$ $\rightarrow$ {\bf (|)} | {\bf\{|\}} | {\bf;} | {\bf,}\\
$Operator$ $\rightarrow$ {\bf =} | {\bf  +} | {\bf -} | {\bf *} | {\bf /} | {\bf <} | {\bf <= } | {\bf> } |

\hspace{1.5cm}|{\bf >=} | {\bf ==} | {\bf !=} | {\bf \&\&} | {\bf ||} | {\bf !}\\

Los reescribiremos de la siguiente manera:

\noindent$InputElement$ $\rightarrow$ $WhiteSpace$ | $Comment$ | $Token$ \\
$WhiteSpace$ $\rightarrow$ {\bf space} | $\backslash$ t | $\backslash$r | $\backslash$n | $\backslash$f | $\backslash$r$\backslash$n \\
$Comment$ $\rightarrow$ //.*\$ \\
$Token$ $\rightarrow$ $Identifier$ | $Keyword$ | $Literal$ | $Separator$ | $Operator$ \\
$Identifier$ $\rightarrow$ $Letter$ | $Identifier$$Letter$ | $Identifier$$Digit$ \\
$Letter$ $\rightarrow$ [a-zA-Z]\\
$Digit$ $\rightarrow$ [0-9] \\
$Keyword$ $\rightarrow$ {\bf boolean} | {\bf else} | {\bf if} | {\bf int} | {\bf main} | {\bf void} | {\bf while} \\
$Literal$ $\rightarrow$ $Boolean$ | $Integer$ \\ 
$Boolean$ $\rightarrow$ {\bf true} | {\bf false}\\
$Integer$ $\rightarrow$ $Digit$ | $Integer$$Digit$ \\
$Separator$ $\rightarrow$ {\bf (|)} | {\bf\{|\}} | {\bf;} | {\bf,}\\
$Operator$ $\rightarrow$ {\bf =} | {\bf  +} | {\bf -} | {\bf *} | {\bf /} | {\bf <} | {\bf <= } | {\bf> } |

\hspace{1.5cm}              | {\bf >=} | {\bf ==} | {\bf !=} | {\bf \&\&} | {\bf ||} | {\bf !}\\


Luego eliminamos las recursiones usando $^{*}$ o $^{+}$ y factorizamos utilizando $?$ entonces Integer va a ser:

\noindent$Integer$ $\rightarrow$ $[0-9]^{+}$ \\
$Identifier$ $\rightarrow$ $[a-zA-Z][a-zA-Z0-9]^{*}$ \\
$Operator$ $\rightarrow$ {\bf =}({\bf =})? | {\bf  +} | {\bf -} | {\bf *} | {\bf /} | {\bf <}({\bf =})$?$ |

\hspace{1.5cm}              | {\bf >}({\bf =})$?$ | {\bf !}({\bf =})? | {\bf \&\&} | {\bf ||}\\


Quedando la siguiente gramatica

\noindent$InputElement$ $\rightarrow$ $WhiteSpace$ | $Comment$ | $Token$ \\
$WhiteSpace$ $\rightarrow$ {\bf space} | $\backslash$ t | $\backslash$r | $\backslash$n | $\backslash$f | $\backslash$r$\backslash$n \\
$Comment$ $\rightarrow$ //.*\$ \\
$Token$ $\rightarrow$ $Identifier$ | $Keyword$ | $Literal$ | $Separator$ | $Operator$ \\
$Identifier$ $\rightarrow$ $[a-zA-Z]([a-zA-Z0-9])^{*}$ \\
$Keyword$ $\rightarrow$ {\bf boolean} | {\bf else} | {\bf if} | {\bf int} | {\bf main} | {\bf void} | {\bf while} \\
$Literal$ $\rightarrow$ $Boolean$ | $Integer$ \\ 
$Boolean$ $\rightarrow$ {\bf true} | {\bf false}\\
$Integer$ $\rightarrow$ $[0-9]^{+}$ \\
$Separator$ $\rightarrow$ {\bf (|)} | {\bf\{|\}} | {\bf;} | {\bf,}\\
$Operator$ $\rightarrow$ {\bf =}({\bf =})? | {\bf  +} | {\bf -} | {\bf *} | {\bf /} | {\bf <}({\bf =})$?$ |

\hspace{1.5cm}              | {\bf >}({\bf =})$?$ | {\bf !}({\bf =})? | {\bf \&\&} | {\bf ||}\\


\subsection{Transformación de la gramatica de la sintaxis de Jay}


La idea ahora es realizar la misma acción para la gramática que define la sintaxis de Jay:

\noindent$Program$ $\rightarrow$ {\bf void main ( )} {\bf \{}$Declarations$ $Statements${\bf \} } \\
$Declarations$ $\rightarrow$ $\lambda$ | $Declarations$ $Declaration$\\
$Declaration$ $\rightarrow$ $Type$ $Indentifiers$ {\bf ;}\\
$Type$ $\rightarrow$ {\bf int} | {\bf boolean}\\
$Identifiers$ $\rightarrow$ $Identifier$ | $Identifiers${\bf ,}$Identifier$\\
$Statements$ $\rightarrow$ $\lambda$ | $Statements$ $Statement$\\
$Statement$ $\rightarrow$ {\bf ;} | $Block$ | $Assignment$ | $IfStatement$ | $WhileStatement$\\
$Block$ $\rightarrow$ {\bf \{}$Statements${\bf \}}\\
$Assignment$ $\rightarrow$ $Identifier$ {\bf =} $Expression$ {\bf ;}\\
$IfStatement$ $\rightarrow$ {\bf if(} $Expression$ {\bf )} $Statement$ | {\bf if(} $Expression$ {\bf )} $Statement$ {\bf else} $Statement$\\
$WhileStatement$ $\rightarrow$ {\bf while(}$Expression$ {\bf )} $Statement$\\
$Expression$ $\rightarrow$ $Conjuntion$ | $Expression$ {\bf ||} $Conjunction$\\
$Conjunction$ $\rightarrow$ $Relation$ | $Conjunction$ \&\& $Relation$\\
$Relation$ $\rightarrow$ $Addition$ |

\hspace{1.5cm} |$Relation$ {\bf <} $Addition$ | $Relation$ {\bf <=} $Addition$ |

\hspace{1.5cm} |$Relation$ {\bf >} $Addition$ | $Relation$ {\bf >=} $Addition$ |

\hspace{1.5cm} |$Relation$ {\bf ==} $Addition$ | $Relation$ {\bf !=} $Addition$\\ 
$Addition$ $\rightarrow$  $Term$ | $Addition$ {\bf +} $Term$ | $Addition$ {\bf -} $Term$\\
$Term$ $\rightarrow$ $Negation$ | $Term$ {\bf *} $Negation$ | $Term$ {\bf /} $Negation$\\
$Negation$ $\rightarrow$ $Factor$ | {\bf !} $Factor$\\
$Factor$ $\rightarrow$ $Identifier$ | $Literal$ | {\bf (} $Expression$ {\bf )}\\

Ahora vamos a eliminar las recursiones a izquierda como las de la producion $Declarations$, $Identifiers$, $Statements$, $Expression$, $Conjunction$, $Relation$, $Addition$, $Term$:

\noindent$Declarations$ $\rightarrow$ $(Declaration)^{*}$\\
$Identifiers$ $\rightarrow$ $Identifier$ ({\bf ,}$Identifier)^{*}$\\
$Statements$ $\rightarrow$ $Statement^{*}$\\
$Expression$ $\rightarrow$ $Conjuntion$ ({\bf ||} $Conjunction$)$^{*}$\\
$Conjunction$ $\rightarrow$ $Relation$ (\&\& $Relation$)$^{*}$\\
$Relation$ $\rightarrow$ $Addition$ ($Comparison$ $Addition$)$^{*}$\\
$Comparison$ $\rightarrow$ {\bf <} | {\bf <=} | {\bf >} | {\bf >=} | {\bf ==} | {\bf !=}\\
$Addition$ $\rightarrow$  $Term$ ($Sums$ $Term$)$^{*}$\\
$Sums$ $\rightarrow$ {\bf +} | {\bf -}\\
$Term$ $\rightarrow$ $Negation$ ($Mult$ $Negation$)$^{*}$\\
$Mult$ $\rightarrow$ {\bf *} | {\bf /}\\

Note que para poder extender la producción $Relation$, $Addition$ y $Term$ tuvimos que agregar tres producciones auxiliares, $Comparison$, $Sums$ y $Mult$ respectivamente. Además como $Declarations$ y $Statements$ llaman a otra produccion ninguna o muchas veces, vamos a eliminar estas dos producciones y colocar en su reemplazo el lado derecho del mismo.
Pasemos ahora a colocar los $?$ que es otro caracter que extiende la gramatica original y me va a reducir un poco mas lo que ya tengo. Agregaremos este símbolo en las siguientes producciones:

\noindent$IfStatement$ $\rightarrow$ {\bf if(} $Expression$ {\bf )} $Statement$ ({\bf else} $Statement$)$?$\\
$Negation$ $\rightarrow$ ({\bf !})$?$ $Factor$\\
$Comparison$ $\rightarrow$ {\bf <}({\bf =})$?$ | {\bf >}({\bf =})$?$  | {\bf ==} | {\bf !=}\\


Veamos como quedan todos estos cambios al ponerlos todos juntos

\noindent$Program$ $\rightarrow$ {\bf void main ( )} {\bf \{}$Declaration^{*}$ $Statement^{*}${\bf \} } \\
$Declaration$ $\rightarrow$ $Type$ $Indentifiers$ {\bf ;}\\
$Type$ $\rightarrow$ {\bf int} | {\bf boolean}\\
$Identifiers$ $\rightarrow$ $Identifier$ ({\bf ,}$Identifier)^{*}$\\
$Statement$ $\rightarrow$ {\bf ;} | $Block$ | $Assignment$ | $IfStatement$ | $WhileStatement$\\
$Block$ $\rightarrow$ {\bf \{}$Statement^{*}${\bf \}}\\
$Assignment$ $\rightarrow$ $Identifier$ {\bf =} $Expression$ {\bf ;}\\
$IfStatement$ $\rightarrow$ {\bf if(} $Expression$ {\bf )} $Statement$ ({\bf else} $Statement$)$?$\\
$WhileStatement$ $\rightarrow$ {\bf while(}$Expression$ {\bf )} $Statement$\\
$Expression$ $\rightarrow$ $Conjuntion$ ({\bf ||} $Conjunction$)$^{*}$\\
$Conjunction$ $\rightarrow$ $Relation$ (\&\& $Relation$)$^{*}$\\
$Relation$ $\rightarrow$ $Addition$ ($Comparison$ $Addition$)$^{*}$\\
$Comparison$ $\rightarrow$ {\bf <}({\bf =})$?$ | {\bf >}({\bf =})$?$  | {\bf ==} | {\bf !=}\\
$Addition$ $\rightarrow$  $Term$ ($Sums$ $Term$)$^{*}$\\
$Sums$ $\rightarrow$ {\bf +} | {\bf -}\\
$Term$ $\rightarrow$ $Negation$ ($Mult$ $Negation$)$^{*}$\\
$Mult$ $\rightarrow$ {\bf *} | {\bf /}\\
$Negation$ $\rightarrow$ ({\bf !})$?$ $Factor$\\
$Factor$ $\rightarrow$ $Identifier$ | $Literal$ | {\bf (} $Expression$ {\bf )}\\


Lo que nosotros elegimos como herramienta para hacer el trabajo práctico fue usar Javacc en el lenguaje Java. Al
tener esta gramática definida pasamos a definir un archivo Jay.jj que se encargará de leer esta gramatica y darnos el parser
deseado.

\subsection{Traducción dirigida por sintaxis}

Ahora a la gramática vamos a tener que agregarle atributos para obtener luego de parsear todo el código Jay un código HTML.

Para mas detalle de esto, ver el código en Jay.jj.



